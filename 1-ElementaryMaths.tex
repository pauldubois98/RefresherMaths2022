\section{Elementary Maths}

\subsection{Mathematical Objects \& Notations}
\subparagraph{Sets}
\begin{definition}[Sets]
    Unordered list of elements.
\end{definition}
\begin{notation}[Sets]
    $\in$, $\{ \text{True}, \text{False} \}$, $\{ a \mid condition \}$, $\{ a, b, c \dots \}$, $\emptyset$
\end{notation}

\begin{remark}[Russell Paradox]
    \textit{(digression)}\\
    Need to be careful when defining set: some definitions are pathological.
    
    e.g.: Take $Y = \{x \mid x \not\in x\}$:
    $Y \in Y \iff Y \not\in Y$
\end{remark}
\begin{notation}[Usual Sets]
    $\B$, $\N$, $\Z$, $\Q$, $\R$, $\C$, $\N^*$, $\R^+$...
\end{notation}

\subparagraph{Functions}
\begin{definition}[Functions]
    Assignment from a set to another.
\end{definition}
\begin{notation}[Function]
    $f: X \to Y$, $f(x)=blah$, $f: x \mapsto blah$.
\end{notation}
\begin{question}
    Which ones of these function are well-defined ?
    \begin{itemize}
        \item $f:k\in\{0,1,2,3,4\}\mapsto 24/k\in \N$
        \item $f:k\in \{1,2,3,4\}\mapsto 24/k\in \N$
        \item $f:k\in \{1,2,3,4,5\}\mapsto 24/k\in \N$
        \item $f:k\in \{1,2,3,4\}\mapsto k\in \{1,2\}$
        \item $f:k\in \{1,2,3,4\}\mapsto k\in \{1,2,3,4,5\}$
    \end{itemize}
\end{question}

\subparagraph{Quantifiers}
\begin{notation}[$\forall$]
    For all elements in set, e.g.: $\forall x \in \R, x^2 \geq 0$.
\end{notation}
\begin{notation}[$\exists$]
    There exists an element in set, e.g.: $\exists x \in \R \text{ s.t. } x^2 > 1$.
\end{notation}
\begin{notation}[$\exists !$]
    There exists a unique element in set, e.g.: $\exists ! x \in \R \text{ s.t. } x^2 \leq 0$.
\end{notation}
\begin{question}
    \begin{itemize}
        \item Express "all natural numbers are positive" with quantifiers
        \item Express $\forall x \geq 0, \ \sqrt{x} \geq 0$ in a sentence
    \end{itemize}
\end{question}
\begin{definition}[Subset / Inclusion]
    $X \subseteq Y$ if $\forall x \in X, x \in Y$
\end{definition}
\begin{definition}[Disjoint Sets]
    $X$ and $Y$ are disjoint if $\forall x \in X, x \not\in Y$ (or if $\forall y \in Y, y \not\in X$).
\end{definition}

\begin{definition}[Power Set]
    $\Pow{X} = \{ Y \mid Y \subseteq X \}$\\
    e.g.: $\Pow{\{1,2,3\}}=\{ \emptyset, \{1\},\{2\},\{3\}, \{1,2\},\{1,3\},\{2,3\}, \{1,2,3\} \}$
\end{definition}
\begin{definition}[Cartesian Product]
    $X \times Y = \{ (x,y) \mid x \in X, y \in Y \}$\\
    e.g.: $\{a,b\} \times \{1,2,3\} = \{ (a,1),(a,2),(a,3), (b,1),(b,2),(b,3) \}$\\
    Extension: $X_1 \times \dots \times X_n = \prod_{k=1}^n X_k$
\end{definition}



\subsection{Axioms}
Here $ \star $ and $ \dagger $ will operations.
\begin{definition}[Associativity]
    $\star$ is associative if $\forall x,y,z, \ (x \star y) \star z = x \star (y \star z)$
\end{definition}
\begin{definition}[Commutativity]
    $\star$ is associative if $\forall x,y, \ (x \star y) = y \star x$
\end{definition}
\begin{definition}[Identity]
    $1_{\star}$ is identity for $\star$ if $\forall x, \ 1_{\star} \star x = x \star 1_{\star} = x$
\end{definition}
\begin{definition}[Annihilator]
    $0_{\star}$ is annihilator for $\star$ if $\forall x, \ 0_{\star} \star x = x \star 0_{\star} = 0_{\star}$
\end{definition}
\begin{definition}[Distributivity]
    $\star$ is distributive over $\dagger$ if $\forall x,y,z \ x \star (y \dagger z) = (x \star y) \dagger (x \star z)$
\end{definition}


 of $\land$ over $\lor$:  $x \land (y \lor z) = (x \land y) \lor (x \land z)$
\begin{question}
    \begin{itemize}\textit{(make a table)}
        \item Which of these are commutative: addition, subtraction, multiplication, division, power?
        \item Which of these are associative: addition, subtraction, multiplication, division, power?
        \item What is identity for: addition, subtraction, multiplication, division, power?
        \item What is annihilator for: addition, subtraction, multiplication, division, power?
    \end{itemize}
\end{question}
\begin{question}
    \begin{itemize}
        \item Think of an operation that is commutative, but not associative
        \item Think of an operation that is associative, but not commutative
    \end{itemize}
\end{question}






\subsection{Objects \& Notations}

- $\N$, $\Z$, $\Q$, $\R$
- scalars vs vectors
- logic \& booleans
\subsection{Proofs}
Example of proofs and non-proofs
- direct
- splitting cases
- induction
- contradiction
\subsection{Geometry}
- equations of lines/planes, etc... => vectors / scalar \& equation manipulations
\subsection{Sets}
- min/max \& sup/inf => start using for all / there exists
\subsection{Integers}
- prime numbers (infinite nb by Euclide)
- unique factorization 
- finding primes between 1 and 100 => time complexity of algo?
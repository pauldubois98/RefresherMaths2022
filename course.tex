\documentclass[11pt,a4paper]{article}

%format
\usepackage[utf8]{inputenc}
\usepackage[T1]{fontenc}
\usepackage[english]{babel}
\usepackage[margin=2.5cm]{geometry}
%math
\usepackage{amsthm}
\usepackage{amsmath}
\usepackage{amsfonts}
\usepackage{amssymb}
\usepackage{stmaryrd}
\usepackage{nicefrac}
\usepackage{mathtools}
%others
\usepackage{hyperref}
\usepackage{graphicx}
\usepackage{tikz}

%environments
\newtheorem*{remark}{Remark}
\newtheorem*{notation}{Notation}
\newtheorem*{definition}{Definition}
\newtheorem*{example}{Example}
\newtheorem*{question}{Question}
\newtheorem*{exercise}{Exercise}
\newtheorem*{proposition}{Proposition}
\newtheorem*{property}{Property}
\newtheorem{lemma}{Lemma}[section]
\newtheorem{theorem}{Theorem}[section]
\newtheorem{corollary}{Corollary}[section]
\newtheorem{conjecture}{Conjecture}[section]
%commands
%\newcommand{\name}[num]{definition}
\newcommand{\primes}{\mathbb{P}}
%\newcommand{\P}{\mathbb{P}}
\newcommand{\N}{\mathbb{N}}
\newcommand{\Z}{\mathbb{Z}}
\newcommand{\Q}{\mathbb{Q}}
\newcommand{\D}{\mathbb{D}}
\newcommand{\R}{\mathbb{R}}
\newcommand{\C}{\mathbb{C}}
\newcommand{\F}{\mathbb{F}}
\newcommand{\B}{\mathbb{B}}
\newcommand{\Norm}[2][]{\text{Norm}_{#1}(#2)}
\newcommand{\norm}[2][]{\left\lVert #2 \right\rVert_{#1}}
\newcommand{\inner}[2]{\left\langle #1,#2 \right\rangle}
\newcommand{\floor}[1]{\lfloor #1 \rfloor}
\newcommand{\ceil}[1]{\lceil #1 \rceil}
\newcommand{\abs}[1]{| #1 |}
\newcommand{\card}[1]{| #1 |}
\newcommand{\curt}[1]{\sqrt[3]{#1}}
\newcommand{\Ker}[1]{\text{Ker}(#1)}
\newcommand{\Image}[1]{\text{Im}(#1)}
\newcommand{\Rank}[1]{\text{Rank}(#1)}
\newcommand{\Nullity}[1]{\text{Nullity}(#1)}
\newcommand{\Span}[1]{\text{Span}(#1)}
\newcommand{\Trace}[1]{\text{Tr}(#1)}
\newcommand{\Det}[1]{\text{Det}(#1)}
\newcommand{\degree}[1]{\partial #1}
\newcommand{\Pow}[1]{\mathcal{P}(#1)}

\title{Refresher Maths Course}
\author{Paul Dubois}
\date{September 2022}

\begin{document}
\maketitle
\begin{abstract}
    This course teaches basic mathematical methodologies for proofs.
    It is intended for students with a lack of mathematical background, or with a lack of confidence in mathematics.
    We will try to cover most of the prerequisites of the courses in the Master’s, i.e. basic algebra/analysis and basic application.
\end{abstract}
\setcounter{tocdepth}{5}
\tableofcontents
\newpage

\setcounter{section}{-1}
\section{Introduction}
Hello! welcome to this maths refresher course for DSBA 2022! This is the best course ever!
\paragraph{Presentation}
\begin{itemize}
    \item Paul Dubois, PhD Student @ Centrale, end of 1st year
    \item Email: \href{mailto:b00795695@essec.edu}{b00795695@essec.edu} (for any question), answer within 1 working day
\end{itemize}
\paragraph{Course Format}
\subparagraph{Lectures}
\begin{itemize}
    \item 8*3h arranged as 1h20min lecture - $\nicefrac{1}{3}$h break - 1h20min lecture
    \item No pb class planned, but lectures will have integrated live exercises
    \item Interrupt if needed (but may also ask at the end of the lecture)
\end{itemize}
\subparagraph{Examination}
\begin{itemize}
    \item Course is pass/fail
    \item Most (in fact hopefully all) of you will pass
    \item There will be sets of exercises (about one per lecture), it is advised to attempt it all (only the starred questions will be compulsory)
    \item As the goal is to learn, you will be able to resubmit exercise sets, but you will lose $10\%$ every-time you re-submit (so that you have some incentive to try your best the 1st time)
    \item Best $\nicefrac{(n-1)}{n}$ count, need average $\geq 70 \%$ to pass
    \item In the unlikely event of not passing, you will be able to do some extra work to pass
\end{itemize}
\paragraph{Questions?}
\newpage

\section{Elementary Maths}
\subsection{Objects \& Notations}
- set notation
- function notations
- $\N$, $\Z$, $\Q$, $\R$
- scalars vs vectors
- logic \& booleans
\subsection{Proofs}
Example of proofs and non-proofs
- direct
- splitting cases
- induction
- contradiction
\subsection{Geometry}
- equations of lines/planes, etc... => vectors / scalar \& equation manipulations
\subsection{Sets}
- min/max \& sup/inf => start using for all / there exists
\subsection{Integers}
- prime numbers (infinite nb by Euclide)
- unique factorization 
- finding primes between 1 and 100 => time complexity of algo?
\section{Sizes of infinity}
Little digression on sets cardinalities.
\url{https://drive.google.com/file/d/1_Yb8hH40KMac3xvwTG6WLkYiQooIaSOV/view?usp=sharing}
\section{Complex numbers}
argand diagram

\section{Asymptotic analysis (limits)}
- def of sequence: recursive and general form
- usual sequences (arithmetic/geometric)
- convergence of sequences
\section{Infinite \& partial sums}
- sum of sequences
- sum of usual (arithmetic/geometric) sequences
- def of series
- convergence of series

\section{Functions \& Inverses}
finding roots \& inverses
\section{Usual functions}
- plot \& limit behaviour of: polynomials, exp, log, sin, cos, tan, sinh, cosh, tanh, arccos, arcsin, arctan
\section{Differentiation}
- from scratch
- derivatives of usual functions
- chain-law \& co

\section{Integration}
- from scratch (area under curve, taking limit of rectangles)
- antiderivative (do proof?)
- integral of usual functions
- integration by part? (if time!)
- integration by substitution? (if time!)
\section{Taylor series}
- theory \& practice
- usual Taylor expansions
- example of convergence
\section{Fourier series? (if not late!)}
\section{Differential calculus? (if not late!)}

\section{Vector spaces}
- def of vect sp
- norm
- basic propr
\section{Matrices}
- def
- linear mapping of vect sp
- inverse: def, existance (det), finding inverse
- rank \& kernel
- eigenvalues

\section{Non-linear multi-dimensional functions}
- eg: cost func
- partial derivatives
- gradient
- convexity?
- optim: gradient descent
\section{Regressions}
- by hand
- theory
- non linear

\section{PCA? (if time)}
\section{Basis of ML (perceptron)? (if time)}



\end{document}
\documentclass[11pt,a4paper]{article}

%format
\usepackage[utf8]{inputenc}
\usepackage[T1]{fontenc}
\usepackage[english]{babel}
\usepackage[margin=2.5cm]{geometry}
%math
\usepackage{amsthm}
\usepackage{amsmath}
\usepackage{amsfonts}
\usepackage{amssymb}
\usepackage{stmaryrd}
\usepackage{nicefrac}
\usepackage{mathtools}
%others
\usepackage{hyperref}
\usepackage{graphicx}
\usepackage{tikz}

%environments
\newtheorem*{remark}{Remark}
\newtheorem*{notation}{Notation}
\newtheorem*{definition}{Definition}
\newtheorem*{example}{Example}
\newtheorem*{question}{Question}
\newtheorem*{exercise}{Exercise}
\newtheorem*{proposition}{Proposition}
\newtheorem*{property}{Property}
\newtheorem{lemma}{Lemma}[section]
\newtheorem{theorem}{Theorem}[section]
\newtheorem{corollary}{Corollary}[section]
\newtheorem{conjecture}{Conjecture}[section]
%commands
%\newcommand{\name}[num]{definition}
\newcommand{\primes}{\mathbb{P}}
%\newcommand{\P}{\mathbb{P}}
\newcommand{\N}{\mathbb{N}}
\newcommand{\Z}{\mathbb{Z}}
\newcommand{\Q}{\mathbb{Q}}
\newcommand{\D}{\mathbb{D}}
\newcommand{\R}{\mathbb{R}}
\newcommand{\C}{\mathbb{C}}
\newcommand{\F}{\mathbb{F}}
\newcommand{\B}{\mathbb{B}}
\newcommand{\Norm}[2][]{\text{Norm}_{#1}(#2)}
\newcommand{\norm}[2][]{\left\lVert #2 \right\rVert_{#1}}
\newcommand{\inner}[2]{\left\langle #1,#2 \right\rangle}
\newcommand{\floor}[1]{\lfloor #1 \rfloor}
\newcommand{\ceil}[1]{\lceil #1 \rceil}
\newcommand{\abs}[1]{| #1 |}
\newcommand{\card}[1]{| #1 |}
\newcommand{\curt}[1]{\sqrt[3]{#1}}
\newcommand{\Ker}[1]{\text{Ker}(#1)}
\newcommand{\Image}[1]{\text{Im}(#1)}
\newcommand{\Rank}[1]{\text{Rank}(#1)}
\newcommand{\Nullity}[1]{\text{Nullity}(#1)}
\newcommand{\Span}[1]{\text{Span}(#1)}
\newcommand{\Trace}[1]{\text{Tr}(#1)}
\newcommand{\Det}[1]{\text{Det}(#1)}
\newcommand{\degree}[1]{\partial #1}
\newcommand{\Pow}[1]{\mathcal{P}(#1)}

\title{Refresher Maths Course}
\author{Paul Dubois}
\date{September 2022}

\begin{document}
\maketitle
\begin{abstract}
    This course teaches basic mathematical methodologies for proofs.
    It is intended for students with a lack of mathematical background, or with a lack of confidence in mathematics.
    We will try to cover most of the prerequisites of the courses in the Master’s, i.e. basic algebra/analysis and basic application.
\end{abstract}
\setcounter{tocdepth}{5}
\tableofcontents
\newpage

\setcounter{section}{-1}
\section{Introduction}
Hello! welcome to this maths refresher course for DSBA 2022! This is the best course ever!
\paragraph{Presentation}
\begin{itemize}
    \item Paul Dubois, PhD Student @ Centrale, end of 1st year
    \item Email: \href{mailto:b00795695@essec.edu}{b00795695@essec.edu} (for any question), answer within 1 working day
\end{itemize}
\paragraph{Course Format}
\subparagraph{Lectures}
\begin{itemize}
    \item 8*3h arranged as 1h20min lecture - $\nicefrac{1}{3}$h break - 1h20min lecture
    \item No pb class planned, but lectures will have integrated live exercises
    \item Interrupt if needed (but may also ask at the end of the lecture)
\end{itemize}
\subparagraph{Examination}
\begin{itemize}
    \item Course is pass/fail
    \item Most (in fact hopefully all) of you will pass
    \item There will be sets of exercises (about one per lecture), it is advised to attempt it all (only the starred questions will be compulsory)
    \item As the goal is to learn, you will be able to resubmit exercise sets, but you will lose $10\%$ every-time you re-submit (so that you have some incentive to try your best the 1st time)
    \item Best $\nicefrac{(n-1)}{n}$ count, need average $\geq 70 \%$ to pass
    \item In the unlikely event of not passing, you will be able to do some extra work to pass
\end{itemize}
\paragraph{Questions?}
\newpage

\section{Elementary Maths}
Should be fairly easy, this section is just so that everyone is one the same page and use the same notation for the rest of the course.
I will go fast as I assume you have already seen this before.
\subsection{Mathematical Objects \& Notations}
\subparagraph{Sets}
\begin{definition}[Sets]
    Unordered list of elements.
\end{definition}
\begin{notation}[Sets]
    $\in$, $\{ \text{True}, \text{False} \}$, $\{ a \mid condition \}$, $\{ a, b, c \dots \}$, $\emptyset$
\end{notation}

\begin{remark}[Russell Paradox]
    \textit{(digression)}\\
    Need to be careful when defining set: some definitions are pathological.
    
    e.g.: Take $Y = \{x \mid x \not\in x\}$:
    $Y \in Y \iff Y \not\in Y$
\end{remark}
\begin{notation}[Usual Sets]
    $\B$, $\N$, $\Z$, $\Q$, $\R$, $\C$, $\N^*$, $\R^+$...
\end{notation}

\subparagraph{Functions}
\begin{definition}[Functions]
    Assignment from a set to another.
\end{definition}
\begin{notation}[Function]
    $f: X \to Y$, $f(x)=blah$, $f: x \mapsto blah$.
\end{notation}
\begin{question}
    Which ones of these function are well-defined ?
    \begin{itemize}
        \item $f:k\in\{0,1,2,3,4\}\mapsto 24/k\in \N$
        \item $f:k\in \{1,2,3,4\}\mapsto 24/k\in \N$
        \item $f:k\in \{1,2,3,4,5\}\mapsto 24/k\in \N$
        \item $f:k\in \{1,2,3,4\}\mapsto k\in \{1,2\}$
        \item $f:k\in \{1,2,3,4\}\mapsto k\in \{1,2,3,4,5\}$
    \end{itemize}
\end{question}

\subparagraph{Quantifiers}
\begin{notation}[$\forall$]
    For all elements in set, e.g.: $\forall x \in \R, x^2 \geq 0$.
\end{notation}
\begin{notation}[$\exists$]
    There exists an element in set, e.g.: $\exists x \in \R \text{ s.t. } x^2 > 1$.
\end{notation}
\begin{notation}[$\exists !$]
    There exists a unique element in set, e.g.: $\exists ! x \in \R \text{ s.t. } x^2 \leq 0$.
\end{notation}
\begin{question}
    \begin{itemize}
        \item Express "all natural numbers are positive" with quantifiers
        \item Express $\forall x \geq 0, \ \sqrt{x} \geq 0$ in a sentence
    \end{itemize}
\end{question}
\begin{definition}[Subset / Inclusion]
    $X \subseteq Y$ if $\forall x \in X, x \in Y$
\end{definition}
\begin{definition}[Disjoint Sets]
    $X$ and $Y$ are disjoint if $\forall x \in X, x \not\in Y$ (or if $\forall y \in Y, y \not\in X$).
\end{definition}

\begin{definition}[Power Set]
    $\Pow{X} = \{ Y \mid Y \subseteq X \}$\\
    e.g.: $\Pow{\{1,2,3\}}=\{ \emptyset, \{1\},\{2\},\{3\}, \{1,2\},\{1,3\},\{2,3\}, \{1,2,3\} \}$
\end{definition}
\begin{definition}[Cartesian Product]
    $X \times Y = \{ (x,y) \mid x \in X, y \in Y \}$\\
    e.g.: $\{a,b\} \times \{1,2,3\} = \{ (a,1),(a,2),(a,3), (b,1),(b,2),(b,3) \}$\\
    Extension: $X_1 \times \dots \times X_n = \prod_{k=1}^n X_k$
\end{definition}



\subsection{Axioms}
Here $ \star $ and $ \dagger $ will operations.
\begin{definition}[Associativity]
    $\star$ is associative if $\forall x,y,z, \ (x \star y) \star z = x \star (y \star z)$
\end{definition}
\begin{definition}[Commutativity]
    $\star$ is associative if $\forall x,y, \ (x \star y) = y \star x$
\end{definition}
\begin{definition}[Identity]
    $1_{\star}$ is identity for $\star$ if $\forall x, \ 1_{\star} \star x = x \star 1_{\star} = x$
\end{definition}
\begin{definition}[Annihilator]
    $0_{\star}$ is annihilator for $\star$ if $\forall x, \ 0_{\star} \star x = x \star 0_{\star} = 0_{\star}$
\end{definition}
\begin{definition}[Distributivity]
    $\star$ is distributive over $\dagger$ if $\forall x,y,z \ x \star (y \dagger z) = (x \star y) \dagger (x \star z)$
\end{definition}


 of $\land$ over $\lor$:  $x \land (y \lor z) = (x \land y) \lor (x \land z)$
\begin{question}
    \begin{itemize}\textit{(make a table)}
        \item Which of these are commutative: addition, subtraction, multiplication, division, power?
        \item Which of these are associative: addition, subtraction, multiplication, division, power?
        \item What is identity for: addition, subtraction, multiplication, division, power?
        \item What is annihilator for: addition, subtraction, multiplication, division, power?
    \end{itemize}
\end{question}
\begin{question}
    \begin{itemize}
        \item Think of an operation that is commutative, but not associative
        \item Think of an operation that is associative, but not commutative
    \end{itemize}
\end{question}


\subsection{Boolean algebra}
\textit{The reason we'll do some is because of it's application to programming, in particular to conditions ('if' blocks and 'while' loops).}
\subparagraph{Basic operators}
\begin{definition}[Conjunction]
    $x \land y = xy$
\end{definition}
\begin{definition}[Intersection]
    $X \cap Y = \{ z \mid (z \in X) \land (z \in Y) \}$
\end{definition}
\begin{remark}[Disjoint Sets and Intersection]
    Disjoint sets have empty intersection.
\end{remark}
\begin{definition}[Disjunction]
    $x \lor y = \min(x+y,1)$
\end{definition}
\begin{definition}[Union]
    $X \cup Y = \{ z \mid (z \in X) \lor (z \in Y) \}$
\end{definition}
\begin{definition}[Negation]
    $\lnot: 0,1 \mapsto 1,0$
\end{definition}
\begin{definition}[Set minus / Complement]
    $X \setminus Y = \{ x \in X \mid \lnot (x \in Y) \}$
\end{definition}
[Draw diagrams]
\begin{question}
    Selecting points outside a given region.
\end{question}
\subparagraph{Basic properties}
\begin{property}[Boolean algebra matching ordinary algebra]
    Same laws as ordinary algebra when one matches up $\lor$ with addition and $\land$ with multiplication.
    \begin{itemize}
        \item Associativity of $\lor$: $x \lor (y \lor z) = (x \lor y) \lor z$
        \item Associativity of $\land$: $x \land (y \land z) = (x \land y) \land z$
        \item Commutativity of $\lor$: $x \lor y  = y \lor x$
        \item Commutativity of $\land$: $x \land y  = y \land x$
        \item Distributivity of $\land$ over $\lor$:  $x \land (y \lor z) = (x \land y) \lor (x \land z)$
        \item $0$ is identity for $\lor$: $x \lor 0  = x$
        \item $1$ is identity for $\land$: $x \land 1  = x$
        \item $0$ is annihilator for $\land$: $x \land 0  = 0$
    \end{itemize}
\end{property}
\begin{property}[Boolean algebra specific properties]
    The following laws hold in Boolean algebra, but not in ordinary algebra: 
    \begin{itemize}
        \item Idempotence of $\lor$: $x \lor x = x$
        \item Idempotence of $\land$: $x \land x = x$
        \item Absorption of $\lor$ over $\land$: $x \lor (x \land y)  = x \land y$
        \item Absorption of $\land$ over $\lor$: $x \land (x \lor y)  = x \lor y$
        \item Distributivity of $\lor$ over $\land$:  $x \lor (y \land z) = (x \lor y) \land (x \lor z)$
        \item $1$ is annihilator for $\lor$: $x \lor 1 = 1$
    \end{itemize}
\end{property}
\begin{property}[De Morgan Laws]
    $\lnot (x \land y) = \lnot x \lor \lnot y$
    and
    $\lnot (x \lor y) = \lnot x \land \lnot y$
\end{property}
\begin{proof}
    Truth-tables; prove De Morgan, others as exercise (or just believe me)
\end{proof}

\subparagraph{Other operators}
\begin{definition}[Exclusive Or]
    $x \oplus y$
\end{definition}
\begin{definition}[Implication]
    $x \implies y$
\end{definition}
\begin{property}[Implication and Inclusion]
    If $\forall x \in X, P_1(x) \implies P_2(x)$, then $\{ x \in X \mid P_1(x) \} \subset \{ x \in X \mid P_2(x) \}$.
\end{property}
\begin{proof}
    Trivial.
\end{proof}
\begin{definition}[If and only if]
    $x \iff y$
\end{definition}
\begin{question}
    Express in terms of and, or, not:
    \begin{itemize}
        \item $\oplus$
        \item $\implies$
        \item $\impliedby$
        \item $\iff$
    \end{itemize}
    Write 1st and 2nd digit of addition of 3 binary numbers $a$, $b$, $c$.
\end{question}

\subparagraph{Negation of quantified propositions}
\begin{property}[Negation of $\forall$]
    $\mathrm{not}(\forall x\in X, P(x)) = \exists x\in X, \mathrm{not}(P(x))$
\end{property}
\begin{property}[Negation of $\exists$]
    $\mathrm{not}(\exists x\in X, P(x)) = \forall x\in X, \mathrm{not}(P(x))$
\end{property}
\begin{notation}[Quantifiers and the empty set]
    $\forall x \in \emptyset, \ \dots$ is true ;
    $\exists x \in \emptyset, \ \dots$ is false
\end{notation}
\begin{question} Negate the following
    \begin{itemize}
        \item $\forall x \in \R, \ \exists n \in \N \text{ s.t. } n > x$
        \item ($x_n \to x$): $\forall \epsilon>0, \exists N \in \N \text{ s.t. } \forall n>N, \abs{x_n-x}<\epsilon$
    \end{itemize}
\end{question}



\subsection{Objects \& Notations}
- $\N$, $\Z$, $\Q$, $\R$
- scalars vs vectors
\subsection{Proofs}
Example of proofs and non-proofs
- direct
- splitting cases
- induction
- contradiction
\subsection{Geometry}
- equations of lines/planes, etc... => vectors / scalar \& equation manipulations
\subsection{Sets}
- min/max \& sup/inf => start using for all / there exists
\subsection{Integers}
- prime numbers (infinite nb by Euclide)
- unique factorization 
- finding primes between 1 and 100 => time complexity of algo?
\section{Sizes of infinity}
Little digression on sets cardinalities.

Link for the slides:\\
\url{https://drive.google.com/file/d/1_Yb8hH40KMac3xvwTG6WLkYiQooIaSOV/view?usp=sharing}

Summary:\\
Although $\N \subseteq \Z \subseteq \Q \subseteq \R$, it is in fact the case that $\card{\N} = \card{\Z} = \card{\Q} < \card{\R}$.

\section{Complex numbers}
argand diagram

\section{Asymptotic analysis (limits)}
- def of sequence: recursive and general form
- usual sequences (arithmetic/geometric)
- convergence of sequences
\section{Infinite \& partial sums}
- sum of sequences
- sum of usual (arithmetic/geometric) sequences
- def of series
- convergence of series

\section{Functions \& Inverses}
finding roots \& inverses
\section{Usual functions}
- plot \& limit behaviour of: polynomials, exp, log, sin, cos, tan, sinh, cosh, tanh, arccos, arcsin, arctan
\section{Differentiation}
- from scratch
- derivatives of usual functions
- chain-law \& co

\section{Integration}
- from scratch (area under curve, taking limit of rectangles)
- antiderivative (do proof?)
- integral of usual functions
- integration by part? (if time!)
- integration by substitution? (if time!)
\section{Taylor series}
- theory \& practice
- usual Taylor expansions
- example of convergence
\section{Fourier series? (if not late!)}
\section{Differential calculus? (if not late!)}

\section{Vector spaces}
- def of vect sp
- norm
- basic propr
\section{Matrices}
- def
- linear mapping of vect sp
- inverse: def, existance (det), finding inverse
- rank \& kernel
- eigenvalues

\section{Non-linear multi-dimensional functions}
- eg: cost func
- partial derivatives
- gradient
- convexity?
- optim: gradient descent
\section{Regressions}
- by hand
- theory
- non linear

\section{PCA? (if time)}
\section{Basis of ML (perceptron)? (if time)}



\end{document}